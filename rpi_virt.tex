By setting up a virtual Raspberry Pi, we can further test our software without physically accessing the Raspberry Pi and burning disk images all the time, even after all automatic tests are passed.

Raspberry Pi uses an ARM processor, so the only tool available to emulate it on PCs is \texttt{qemu}.

Basically we follow the instructions from \url{http://www.linux-mitterteich.de/fileadmin/datafile/papers/2013/qemu_raspiemu_lug_18_sep_2013.pdf} but use the latest Raspbian image instead, which can be downloaded from \url{https://www.raspberrypi.org/downloads/}.

And download \texttt{kernel-qemu} from \url{http://web.archive.org/web/20150214035104/http://www.xecdesign.com/downloads/linux-qemu/kernel-qemu} because the original page is unavailable.

Currently this solution has several drawbacks, mostly notable ones are low speed and lack of serial port support.

\vspace{\baselineskip}

Note:

\begin{itemize}
\item The second line of \texttt{/etc/udev/rules.d/90-qemu.rules} is indeed \texttt{KERNEL=="sda?", SYMLINK+="mmcblk0p\%n",}, there is a typo in the picture.
\item Although the new RPi 2 model has an ARMv7 processor, Raspbian is still compiled into ARMv6 (for compatibility with old models), so emulating with \texttt{-cpu arm1176} is reasonable.
\end{itemize}