Gradle\footnote{http://gradle.org/} is a build automation tool and simplifies the build process drastically.
Instead of having a readme file that explains how to build the project, building the project is as easy as executing \texttt{./gradlew build} on the command line.

Gradle is configured in the file \texttt{build.gradle}.
Although it looks like a configuration file, it is actually fully functional code written in Groovy\footnote{http://www.groovy-lang.org/}, a programming language based on Java.

We included a Gradle wrapper in the repository.
This file called \texttt{gradlew} (and \texttt{gradle.bat} for windows) download and run the right Gradle version when executed.
It is recommended to use this file instead of using this file instead of any other locally installed Gradle, because it might be another version of Gradle.

The file \texttt{settings.gradle} specifies the gradle project structure by declaring several subprojects.
There is the opendial subproject, which is a copy of the dialog manager OpenDial\footnote{http://www.opendial-toolkit.net/}.
Next we have the sphinx4 subproject, consisting of sphinx4-core and sphinx4-data, which is a copy of the speech recognition software Sphinx4\footnote{https://github.com/cmusphinx/sphinx4}.
These subprojects are all forked on github from the originals, eventually adapted for our purposes and then included as a Git submodule.
Finally we have a subproject called prompts, that is our own making.
It contains example audio prompts for our application, that are used for the acoustic model training and for testing.
