

These installation instructions have been tested on Ubuntu 14.04.

If you have not installed git on your system, you should do this first:
\begin{lstlisting}[language=bash]
sudo apt-get install git
\end{lstlisting}

This program is the version control system we used and helps to keep changes by different developers in sync.
For details see the section on Github.

Now you can download the project files using the \texttt{git clone} command.
Navigate to the folder where you want to place the project files and execute:
\begin{lstlisting}[language=bash]
git clone https://github.com/EllaVator/EllaVator.git --recursive
\end{lstlisting}

This will create a folder named \texttt{EllaVator} inside the current directory and put the downloaded files inside.
Please note that you should use the \texttt{--recursive} flag here in order to download all the nessecary files, including the subprojects.

Now enter the created folder:
\begin{lstlisting}[language=bash]
cd EllaVator
\end{lstlisting}

Our project requires Java 8. If you haven't installed it already, you can do so using the following commands:
\begin{lstlisting}[language=bash]
sudo add-apt-repository ppa:webupd8team/java
sudo apt-get update
sudo apt-get install oracle-java8-installer
\end{lstlisting}

Now you can build or run the project using one of the following commands:
\begin{lstlisting}[language=bash]
./gradlew build
./gradlew run
\end{lstlisting}
