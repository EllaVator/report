It is technically impossible to test Raspi and elevator connection without opening the elevator control panel.
Therefore, for testing we used an old laptop with serial port and a desktop PC.
For testing purposes we wrote several python scripts, pi4j java testing tool and a small RXTX java implementation.
We also used such shell utilities as minicom.
\\

There were 3 testing phases.

1 testing:

We used python pyserial implementation to test serial connection between Raspi, laptop and PC.
After several unsuccessful attempts to use pi4j and GPIO pins connection, we switched to USB serial converter.
Raspi successfully sent the signals via serial port using USB converter.
We also managed to test the serial connection between a laptop and a desktop PC.
Everything worked as expected.
\\

2 testing:

We connected the laptop to the elevator via laptop serial port with a simple serial cord.
Small RXTX serial port utility was used to send the signals from the laptop.
Everything worked correctly and we managed to control the elevator from the laptop by sending the appropriate signals.
\\

3 testing:

We connected Raspi to the elevator using USB serial converter.
The same RXTX serial port utility was used to send the signals to the elevator.
However whatever floor we tried to send the elevator to it moved to floor 1. 
We suspect that Raspi simply does not have enough power to support serial connection to the elevator.
\\

Our next steps:

Find a way how to boost Raspi serial port power from 5V to 12V.
If it fails, drop Raspi completely and use a variant of desktop PC with an SD drive.
