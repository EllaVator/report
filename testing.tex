Software testing is an essential process in modern software development, in particular in the process of quality assurance (QA) of software.
As the name indicates software testing verifies whether the software does what it is supposed to do.
But though invented for this purpose, software testing can accomplish more than that.
Tests also help to maintain the features that have been developed in the past and protects them from being destroyed accidentally by new code.
On top of that tests can help developers to understand or remember what exactly the code does, as tests are generally easier to understand than code.
Tests provide examples of how the existing code is supposed to be used and therefore provide a very useful addition to the documentation.
The most extreme incarnation of testing might be a development process called Test Driven Development.
In this process tests are even written before the actual code.

Generally there at least two types of tests: unit and integration tests.
While unit tests are thought to test small components (e.g. classes) in isolation, integration tests verify how these components work as a whole.

Seeing time constraints and also our lack of experience, we decided not to be as strict with testing, and we put a focus on integration testing.
All tests are placed in the \texttt{test} source set, i.e. under \texttt{src/test}.
There they can be found automatically by gradle when executing \texttt{./gradlew test} in the commandline.
We used TestNG\footnote{http://testng.org/doc/} as our testing framework.
